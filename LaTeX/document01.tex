% 导言区
\documentclass[12pt]{article} %book, report, letter, ctexart

\usepackage{ctex}	% 处理中文

\newcommand{\degree}{^\circ}	%定义新命令
\title{\heiti first}
\author{\kaishu JachinLin}	%一个字体五种属性
\date{\today}

% 正文区(文稿区)
\begin{document}
	\maketitle
	hello!	% 加个空行就可以另起一行

	$ f(x)=3x^2 + x - 1 $.
	$$ f(x)=3x^2 + x - 1 $$.
	
	{\par}
	
	$ C = 90\degree  $.
	\begin{equation}	%equation(用于产生带编号的行间公式)
		AB^2 = BC^2 + AC^2.
	\end{equation}
	
	{\par}
	% 字体设置 (罗马字体、无寸线字体、打字机字体)
	\textrm{Roman Family}
	\textsf{Sans Serif Family}
	\texttt{Typewriter Family}
	
	{\rmfamily Roman Family}
	{\sffamily Sans Serif Family}
	{\ttfamily Typewriter Family}%大括号进行分组
	
	% 字体系列设置 (粗细、宽度)
	\textmd{Medium Series} \textbf{Boldface Series}
	{\mdseries Medium Series} {\bfseries Boldface Series}
	
	% 字体形状 (直立、斜体、伪斜体、小型大写)
	\textup{Upright Shape} \textit{Italic Shape}
	\textsl{Slanted Shape} \textsc{Small Caps Shape}
	
	{\upshape Upright Shape} {\itshape Italic Shape}
	{\slshape Slanted Shape} {\scshape Small caps Shape}
	
	% 中文字体
	{\songti 宋体} \quad {\heiti 黑体} \quad 
	{\fangsong 仿宋} \quad {\kaishu 楷书}
	中文字体的\textbf{粗体}与\textit{斜体}
	
	% 字体大小
	{\tiny hello}\\
	{\scriptsize hello}\\
	{\footnotesize hello}\\
	{\small hello}\\
	{\normalsize hello}\\
	{\large hello}\\
	{\Large hello}\\
	{\LARGE hello}\\
	{\huge hello}\\
	{\Huge hello}\\
	
	% z中文字号设置
	\zihao {5}
	
	%\myfont
	
	{\par}
	
	\tableofcontents
	
	\section{引言}
	\section{实验方法}
	\section{实验结果}
	\subsection{过程}
	\subsection{结论}
	\section{数据}
	
	
	
	
	
	
\end{document}