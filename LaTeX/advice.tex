\documentclass[UTF-8, a4paper]{ctexart}
\usepackage {ctex}
\title{《洛阳市烈士陵园》引入深思}
\author{}
\date{}
\newcommand{\upcite}[1]{\textsuperscript{\textsuperscript{\cite{#1}}}} 
\begin{document}
	\maketitle
	洛阳市烈士陵园始建于1955年,地处老城区道北陵园路一号,占地116.25亩。园内安葬有不同革命时期的烈士,是全国重点烈士纪念建筑物保护单位。单位编制40名,在岗职工29名,设有办公室、后勤财务股、宣传股、绿化队、开发办、管理股6个职能股室。主要烈士纪念建筑物有:洛阳革命烈士事迹纪念馆、烈士墓区、革命烈士纪念碑、纪念碑附碑(烈士英名录墙)、革命烈士骨灰堂、邙山革命公墓。整体规划分为办公区、烈士褒扬区、职工生活区、游览观赏区。
	
	本属于一个神圣之地,值得人们对先烈们的缅怀和景仰之地,但有报道指出洛阳烈士陵园被长期蚕食,8个陵园区6个被改为商业墓地高价出售,1800多座商业墓挤走258位烈士的陵寝,258位烈士的遗骨被集体“坑葬”,叠层掩埋。今年又进行所谓“修缮”,将仅余的2个陵园区中的一个平毁,仅余的250座烈士墓墓体用挖掘机挖掉150座,墓碑砸碎的恶劣事件。1月4日,《人民日报》发表了记者曲昌荣的一篇调查报道“洛阳市回应烈士陵园‘被毁事件’:是修缮改造,不是商业开发”,这是中央级权威媒体首次就此事件发表的报道。曲记者谨慎地在“被毁”二字上加上了引号,这似乎是一种态度。
	
	河南《大河报》的报道表明,那287位解放军烈士原先是分区安葬的,但是“据陵园工作人员介绍,多年前(按照《人民日报》的报道,准确地说是上世纪90年代),287名烈士陵墓被毁,烈士的遗骨被集中掩埋在这座‘副碑’下,因为地方狭小,采用了层层叠压的方法。”,《人民日报》的曲记者对这一点也侧面地证实了“记者在采访中发现,洛阳烈士陵园的商业开发已近15年,除烈士墓一区、二区及纪念碑附碑安置烈士外,其余6个区皆被开发成商业墓地”,所谓“纪念碑附碑安置烈士”是什么时候的事情,是不是将原先分别安葬在6个区的的烈士再次“集中掩埋”,其中采用了“层层叠压的方法”。
	
	烈士陵园才商业化开发是上个世纪90年代开始的,是有关方面“批准”的,缘由是“国家有关部门提出‘提倡通过开展经营创收来弥补国家财政投入的不足’。”,难怪洛阳有关方面及当事人觉得“憋屈”,因为有“国家有关方面提出”的思路。但是,这句“通过开展经营创收来弥补国家财政投入的不足”听起来好生耳熟,如果人们不算健忘,就应该记得各种公共事业“市场化取向”发端的年代,也是那个时期,在“弥补不足”的口号下,卫生、教育莫不是走上此路,今日的结果如何,已经昭然。
	
	还有一个令人感慨的现象,就是入葬者的规格的演进,曲记者报道说,所谓“革命公墓”起先还勉强体现出一点“革命”的味道,对安葬对象做了限定:一是老红军战士;二是建国前参加工作的老同志、老干部和建国后参加工作的县、团级以上干部;三是省级以上劳动模范和全国荣誉称号获得者;四是高级知识分子和著名爱国人士;五是市级以上劳动模范和市级以上荣誉称号获得者。河南省相关方面当时的批复也是基于这个前提的。“洛阳市民政局和河南省民政厅分别在1993年5月22日和6月1日批准了这个请示。”,同时规定了一个入葬的价格,“单价从2500元到8000元不等”。
	
	然而,商业就是商业,市场就是市场,一切原则、底线都屈服在金钱这个标杆之下。在创收思想的指引下,洛阳烈士陵园的门槛也在不断降低,说穿了就是花大钱,有关系就可以进去。
	至此,再辩解什么“不是商业开发”草民觉得已经没有任何意义,把一个好端端的烈士陵园在“开展经营创收”的思路下,搞成按照金钱多寡分等级的商业墓地,商业墓地的豪华、喧闹、媚俗,代替了烈士陵园的庄严、肃穆、崇高,这样的场景,当然会在视觉上、心灵上造成潜移默化的效应。也就是在这样的氛围下,施工人员才对烈士失却了起码的敬重,才用挖掘机野蛮捣毁烈士墓,用装载机的巨轮在烈士的遗骨上碾压(烈士遗骨离地面仅一米),网民们的愤怒,就是基于对商业化冲击道德底线,金钱至上淡化历史传承的这个过程,绝非仅仅针对一时一事。

	对于此,我觉得领导层也应该有个底线,在这一方面上,民众的觉悟也许读比大多数人更清晰,真的有必要进行反思!
	
	
	
\end{document}

