\documentclass[UTF-8, a4paper]{ctexart}
\usepackage {ctex}
\title{《共圆中华民族伟大复兴的中国梦》读书笔记}
\author{}
\date{}
\newcommand{\upcite}[1]{\textsuperscript{\textsuperscript{\cite{#1}}}} 
\begin{document}
	\maketitle
	党的十八大召开后不久,习近平总书记在参观《复兴之路》展览时提出和阐述了“中国梦”。他指出:“每个人都有理想和追求,都有自己的梦想。现在,大家都在讨论中国梦,我以为,实现中华民族伟大复兴,就是中华民族近代以来最伟大的梦想。”从这时起,中国梦开始深入人心,凝集了所有中国人的夙愿,是所有中国人的期盼。
	
	此次是2014年2月18日习近平在北京钓鱼台国宾馆会见中国国民党荣誉主席连战及随访的台湾各界人士发表的讲话,表明了中国梦是两岸共同的梦,两岸同胞必须相互扶持,不分党派,不分阶层,不分宗教,不分地域,都参与到民族复兴的进程中来。充分说明了中国是一个整体,不容分割,中国梦是中国人民的梦。
	
	关于“中国梦是和平、发展、合作、共赢的梦”的论述。中国梦是中国人民的奋斗目标,也是中国的国际形象。中国梦与各国人民追求和平发展的美好梦想相通。中国的发展离不开世界,世界的繁荣稳定也需要中国。中国的发展,是世界和平力量的壮大,是传递友谊的正能量。随着国力不断增强,中国将在力所能及的范围内承担更多国际责任和义务。要同国际社会一道,推动实现持久和平、共同繁荣的世界梦,为人类和平与发展的崇高事业作出新的更大的贡献。
	
	纸上谈兵终将是一纸空文,为实现中华民族伟大复兴,我们必须携起手来,付出于实际行动,将这项光荣而艰巨的任务一代一代华夏儿女完成,任重而道远。在建设特色社会主义前进的道路上会有许多困难与挑战等着我们,只要一步一个脚印、稳扎稳打向前走,积小胜为大胜,积跬步至千里,一定能圆美丽的中国梦!
	
\end{document}
