\documentclass[UTF-8, a4paper]{ctexart}
\usepackage {ctex}
\title{《在庆祝中国共产党成立八十周年大会上的讲话》读书笔记}
\author{}
\date{}
\newcommand{\upcite}[1]{\textsuperscript{\textsuperscript{\cite{#1}}}} 
\begin{document}
	\maketitle
	 在八十年前共产党产生之时,只有区区五十几个党员,面对的是一个灾难沉重的旧中国。而如今,我们中国共产党已经执政五十多年,拥有六千四百多万的党员,中国人民已拥有一个欣欣向荣的社会主义祖国。经过八十多年的实践告诉我们必须坚持马克思主义与中国实际相结合,坚持科学理论的指导,坚定不移的走自己的路;八十周年的实践启示我们,必须始终紧紧依靠人民群众,诚心诚意为人民谋利益,从人民群众中汲取前进的不竭力量;八十周年的时间还启示我们,必须始终自觉地加强和改进党的建设,不断增强党的创造力、凝聚力和战斗力,永葆党的生机和活力。这是中国共产党奋斗八十周年的基本经验。 正是因为中国共产党,我们才能获得反帝反封建的革命斗争、争取民族独立和人民解放、实现振兴中华的伟大使命。 随着国内外形式的深刻变化,我们党要带领全国人民抓住机遇、迎接挑战,胜利完成三大历史任务,我们就必须贯彻和落实“三个代表”要求。“三个代表”要求,是我们党的立党之本、执政之基、力量之源,也是我们在新世纪全面推进党的建设,不断推进理论创新、制度创新和科技创新,不断夺取建设有中国特色社会主义事业新胜利的根本要求。首先,我们党要始终代表中国先进生产力的发展要求,必须努力符合生产力发展的规律,体现不断推动社会生产力的解放和发展的要求,尤其要体现推动先进生产力发展的要求,通过发展生产力不断提高人民群众的生活水平。其次,我们党要始终代表中国先进文化的前进方向,就是党的理论、路线、纲领、方针、政策和各项工作,必须努力体现发展面向现代化、面向世界、面向未来的,民族的科学的大众的社会主义文化的要求,促进全民族思想道德素质和科学文化素质的不断提高,为我国经济发展和社会进步提供精神动力和智力支持。最终,我们党要始终代表中国最广大人民的根本利益,就是党的理论、路线、纲领、方针、政策和各项工作,必须坚持把人民的根本利益作为出发点和归宿,充分发挥人民群众的积极性主动性创造性,在社会不断发展进步的基础上,使人民群众不断获得切实的经济、政治、文化利益。只有这样我们国家才能不断发展,进步,最终取得共同富裕。 
	
	 中国共产党不仅领导我们推进改革开放和社会主义现代化建设,同时中国共产党引导我们前进的方向。正是中国共产党,才有我们中国的繁荣昌盛。中国共产党在社会不断发展的前提下,不断完善、改造自己是自己适应社会发展的潮流。全党人民根据对历史的总结进行探索开辟了适应中国发展的中国特色社会主义的基本道路。我们应该热爱祖国,热爱人民,志存高远,胸怀宽广,在改革开放和现代化建设的广阔舞台上,充分发挥自己的聪明才智,展现自己的人生价值,努力创造无愧于时代和人民的业绩。
	 
	 
	
\end{document}
